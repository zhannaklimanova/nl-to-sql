\section{Introduction}

Text-to-SQL is a critical area of research focused on the automatic translation of natural language questions into SQL queries. It enables applications such as AI-driven database services and question-answering functionality, making it easier for non-experts to interact with database systems. In this project\footnote{The code for this project is accessible at: https://github.com/zhannaklimanova/nl-to-sql}, we evaluate the capabilities of large language models (LLMs) on a novel domain: Cantus Database, a rich collection of metadata from Latin chants, accessible at \url{https://cantusdatabase.org}/.

\subsection{Cantus Database}
Cantus Database, or CantusDB \cite{cantusdatabase}, is a website and database of Latin ecclesiastical chants from manuscripts and other sources around the world. It is actively maintained by developers at McGill University and is used by more than 250 academics and researchers. In Cantus Database, there are three main objects that are searchable and filterable on the site: \textit{Chants}, \textit{Sources}, and \textit{Feasts}. \textit{Chants} are individual pieces of liturgical music that include musical notation, text, and associated metadata. \textit{Sources} refer to manuscripts or early printed books that preserve these chants. \textit{Feasts} are specific liturgical celebrations within the Christian calendar to which chants are assigned and performed.

Cantus Database provides researchers with an effective way to locate specific data, but its search and filtering options are limited to predefined criteria, restricting users. To address this, we propose a solution that uses natural language queries, allowing users to retrieve results without relying on existing filters, offering greater flexibility in accessing information.

\subsection{Large Language Models}
Recent advancements in LLMs have significantly improved the Text-to-SQL task, which translates natural language queries into SQL code. This task is historically challenging due to the complexities of natural language understanding and database schema reasoning, but innovations in LLM architectures and pre-training now enable more efficient and accessible database interactions.

Among the LLMs currently available, three were selected for this research: OpenAI's ChatGPT-o1 (GPT) \cite{openai2024chatgpt}, chosen for its advanced reasoning capabilities; Anthropic's Claude 3.5 Haiku (Claude) \cite{anthropic2023claude}, noted for its efficiency and extensive token limit; and xAI's Grok-2 (Grok) \cite{grok2}, valued for its integration of real-time web knowledge, which is particularly useful for addressing dynamic and less-documented aspects of Cantus Database.

Incorporating LLMs into website search and filtering poses challenges, primarily the high computational costs of training new models. Additionally, changes to database structures often necessitate time-consuming updates to datasets due to the need for fine-tuning or retraining. Considering the added challenge posed by the lack of a large annotated dataset for Cantus Database, this research investigates the ability of modern pre-trained LLMs to tackle complex Text-to-SQL tasks on an unfamiliar database schema without relying on explicit fine-tuning or task-specific training.

By relying solely on inference, this work explores how effectively these models can generate SQL queries using only a natural language query, a database schema, and selected schema content. This approach reduces computational costs and offers a more efficient, sustainable solution.