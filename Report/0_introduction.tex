\section{Introduction}

% Introduction: What is the motivation behind the project? In general terms, what is the
% hypothesis that you test and how do you go about doing so?

Text-to-SQL is a critical area of research focused on the automatic translation of natural language questions into SQL queries. By bridging the gap between non-expert users and database systems, it significantly enhances data accessibility, streamlines data processing, and enables applications such as intelligent database services, automated data analysis, and database-driven question-answering systems. In this project, we evaluate the capabilities of large language models (LLMs) on a novel domain: the Cantus Database, a rich collection of metadata from Latin chants in manuscripts and early printed books, accessible at https://cantusdatabase.org.

Cantus Database is a website and database of Latin ecclesiastical chants from manuscripts and other sources around the world. It is actively maintained by developers at McGill University and is used by more than 250 academics and researchers. In the Cantus Database, there are three main objects that are searchable and filterable on the site: Chants, Sources, and Feasts. Chants are individual pieces of liturgical music, often drawn from Gregorian chant traditions, that include musical notation, text, and associated metadata. Sources refer to manuscripts or early printed books that preserve these chants, providing critical historical context for their transmission and use. Feasts are specific liturgical celebrations within the Christian calendar, such as Christmas or Easter, to which chants are assigned and performed.

The Cantus Database is structured to provide a simple yet effective way for researchers to locate specific data for their projects. However, the current search and filtering options are limited to the predefined criteria available on the site, which can restrict users who need to perform more specialized queries. To address this, we offer a solution that allows users to search the database using natural language questions (NLQs). By allowing users to input queries in their own words, the system can retrieve relevant results without being constrained by existing filters, providing greater flexibility and precision in accessing the desired information.


Specifically, the research explores whether modern pre-trained LLMs, OpenAI’s o1, xAI’s Grok, and Anthropic’s Claude, can address complex Text-to-SQL problems within the Cantus database schema without explicit fine-tuning or task-specific training. By relying solely on inference, this experiment investigates the extent to which the learned parameters of these models enable Text-to-SQL functionality, using only a natural language query, a database schema, and select schema content as inputs.


This evaluation focuses on simple prompting with GPT, Grok, and Claude on a dataset that has never been tested for Text-to-SQL tasks before. By using only natural language queries, the database schema, and select schema content as inputs, we aim to determine whether inference alone can effectively generate SQL queries without fine-tuning.

We explore GPT, Grok, and Claude to address practical limitations. Training new models has a significant environmental impact due to high computational costs. Additionally, in real-world settings, changes to the database structure often require fine-tuning or retraining, which is time-consuming and resource-intensive. To overcome these challenges, we use inference-based evaluation, as the Cantus database lacks a sufficiently large annotated dataset of Text-to-SQL data. This approach reduces costs and avoids retraining, offering a more efficient and sustainable solution.